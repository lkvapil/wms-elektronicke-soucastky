\documentclass[12pt,a4paper]{article}

% Balíčky
\usepackage[utf8]{inputenc}
\usepackage[czech]{babel}
\usepackage[T1]{fontenc}
\usepackage{cmap}
\usepackage{geometry}
\usepackage{listings}
\usepackage{xcolor}
\usepackage{graphicx}
\usepackage{hyperref}
\usepackage{booktabs}
\usepackage{longtable}
\usepackage{fancyhdr}
\usepackage{amsmath}
\usepackage{float}
\usepackage{enumitem}

% Nastavení stránky
\geometry{
    left=2.5cm,
    right=2.5cm,
    top=3cm,
    bottom=3cm,
    headheight=15pt
}

% Nastavení záhlaví a zápatí
\pagestyle{fancy}
\fancyhf{}
\fancyhead[L]{Projektování IS - Semestrální projekt}
\fancyhead[R]{Kvapil, Szetei, Simandl}
\fancyfoot[C]{\thepage}

% Nastavení odkazů
\hypersetup{
    colorlinks=true,
    linkcolor=black,
    urlcolor=blue,
    citecolor=blue
}

% Definice barev pro zvýraznění kódu
\definecolor{codegreen}{rgb}{0,0.6,0}
\definecolor{codegray}{rgb}{0.5,0.5,0.5}
\definecolor{codepurple}{rgb}{0.58,0,0.82}
\definecolor{backcolour}{rgb}{0.95,0.95,0.92}

% Nastavení pro kód
\lstdefinestyle{mystyle}{
    backgroundcolor=\color{backcolour},   
    commentstyle=\color{codegreen},
    keywordstyle=\color{magenta},
    numberstyle=\tiny\color{codegray},
    stringstyle=\color{codepurple},
    basicstyle=\ttfamily\footnotesize,
    breakatwhitespace=false,         
    breaklines=true,                 
    captionpos=b,                    
    keepspaces=true,                 
    numbers=left,                    
    numbersep=5pt,                  
    showspaces=false,                
    showstringspaces=false,
    showtabs=false,                  
    tabsize=2,
    frame=single
}

\lstset{style=mystyle}

% Přejmenování "Listing" na "Skript"
\renewcommand{\lstlistingname}{Skript}

\begin{document}

% Titulní strana
\begin{titlepage}
    \centering
    \vspace*{1cm}
    
    {\huge\bfseries WMS pro správu elektronických součástek\par}
    \vspace{0.5cm}
    {\Large Semestrální projekt\par}
    \vspace{3cm}
    
    {\Large\textbf{Autoři:}\par}
    \vspace{0.3cm}
    {\Large\itshape Bc. Lukáš Kvapil\par}
    {\Large\itshape Bc. Tomáš Szetei\par}
    {\Large\itshape Bc. Tomáš Simandl\par}
    \vspace{3cm}
    
    \begin{flushleft}
    {\large \textbf{Název předmětu:} Projektování IS - EIT21E\par}
    \vspace{0.3cm}
    {\large \textbf{Fakulta:} Provozně ekonomická fakulta\par}
    \vspace{0.3cm}
    {\large \textbf{Zajišťuje:} Katedra informačního inženýrství (PEF)\par}
    \vspace{0.3cm}
    {\large \textbf{Semestr:} LS 2025/2026\par}
    \end{flushleft}
    
    \vfill
    
    {\large Praha, 10. února 2026\par}
    
\end{titlepage}

% Obsah
\tableofcontents
\newpage

% Hlavní část dokumentu

\section{Formulace problému}

\subsection{Popis problémové domény}

Při výrobě elektronických zařízení je nezbytné efektivně spravovat \textbf{BOM (Bill of Materials)} -- seznam všech použitých součástek. Tradiční metody jako ruční zápis nebo Excel tabulky jsou náchylné k chybám, pomalé a neumožňují snadné sledování součástek napříč projekty.

Současná praxe při správě součástek zahrnuje:
\begin{itemize}
    \item Manuální zápis Part Number do tabulek
    \item Obtížnou synchronizaci mezi skladem a projekty
    \item Riziko chyb při přepisování identifikátorů
    \item Časově náročné vyhledávání informací o součástkách
    \item Složitou evidenci skladových míst
\end{itemize}

\subsection{Cíl systému}

\subsubsection{Hlavní cíl}

Vytvořit desktopovou aplikaci \textbf{WMS pro správu elektronických součástek} (Warehouse Management System) pro automatizovanou správu BOM (Bill of Materials), která eliminuje manuální chyby, urychluje proces evidence součástek a poskytuje komplexní přehled o stavu skladu a přiřazení součástek k projektům.

\subsubsection{Dílčí cíle}

\paragraph{C1: Automatizace skenování a evidence}
\begin{itemize}
    \item Implementovat podporu pro Zebra čtečky čárových kódů
    \item Automaticky parsovat QR kódy a extrahovat strukturovaná data
    \item Detekovat duplicity a automaticky sčítat množství
    \item Ukládat kompletní historii skenování s časovými razítky
    \item Redukovat čas zpracování z 30+ sekund (manuální zápis) na $<$ 1 sekundu
\end{itemize}

\paragraph{C2: Integrace s externími službami}
\begin{itemize}
    \item Připojit TME API pro automatické obohacení dat o kategorie
    \item Získávat popisy, technické parametry a ceny součástek
    \item Zajistit offline funkčnost při nedostupnosti API
    \item Implementovat fuzzy matching pro varianty výrobních kódů
\end{itemize}

\paragraph{C3: Správa projektů a organizace}
\begin{itemize}
    \item Umožnit vytváření a správu projektů
    \item Přiřazovat součástky k více projektům současně (M:N vztah)
    \item Zobrazovat přehled součástek pro každý projekt
    \item Sledovat spotřebu součástek na jednotlivých projektech
\end{itemize}

\paragraph{C4: Správa skladových míst}
\begin{itemize}
    \item Implementovat systém skladových lokací s unikátními kódy
    \item Přiřazovat součástky na konkrétní fyzická místa
    \item Generovat a tisknout štítky skladových míst pomocí ZPL
    \item Podporovat Zebra tiskárny pro tisk štítků 2×1 palec
\end{itemize}

\paragraph{C5: Export a reporting}
\begin{itemize}
    \item Exportovat BOM do CSV formátu s timestampy
    \item Podporovat JSON export pro programové zpracování
    \item Generovat přehledy podle projektů a skladových míst
    \item Umožnit import existujících BOM dat
\end{itemize}

\paragraph{C6: Uživatelská přívětivost}
\begin{itemize}
    \item Vytvořit intuitivní GUI založené na PyQt6
    \item Zajistit rychlou odezvu ($<$ 1 sekunda)
    \item Automatické ukládání všech změn
    \item Multiplatformní kompatibilita (macOS, Windows, Linux)
    \item Minimalizovat potřebu školení uživatelů
\end{itemize}

\subsubsection{Měřitelné výstupy}

\begin{table}[H]
\centering
\begin{tabular}{@{}lll@{}}
\toprule
\textbf{Metrika} & \textbf{Před} & \textbf{Po} \\ \midrule
Čas zpracování 1 součástky & 30-60 s & $<$ 1 s \\
Chybovost při zápisu & 5-10\% & $<$ 0.1\% \\
Dostupnost dat o součástkách & Částečná & 100\% \\
Sledování historie & Žádné & Kompletní \\
Export do jiných systémů & Manuální & Automatický \\ \bottomrule
\end{tabular}
\caption{Porovnání před a po implementaci systému}
\end{table}

\subsubsection{Nefunkční cíle}

\begin{itemize}
    \item \textbf{Výkon:} Zpracování QR kódu $<$ 1 sekunda, odezva UI $<$ 100 ms
    \item \textbf{Spolehlivost:} Žádná ztráta dat díky automatickému ukládání
    \item \textbf{Použitelnost:} Intuitivní ovládání, minimální školení
    \item \textbf{Přenositelnost:} Multiplatformní (macOS, Windows, Linux)
    \item \textbf{Rozšiřitelnost:} Snadné přidání nových API nebo funkcí
    \item \textbf{Údržba:} Jasná architektura, separace vrstev, dokumentace
\end{itemize}

\subsection{Klíčové požadavky}

\subsubsection{Funkční požadavky}

\begin{enumerate}[label=FR\arabic*:]
    \item Automatické skenování QR kódů (čtečka funguje jako klávesnice)
    \item Parsování QR kódů ve formátu \texttt{PN=hodnota,MPN=hodnota,QTY=hodnota,...}
    \item Detekce duplicitních součástek a automatické sčítání množství
    \item Ukládání scan historie pro každou součástku s timestampy
    \item Přiřazování součástek k projektům (M:N vztah)
    \item Přiřazování součástek na skladová místa (M:1 vztah)
    \item Generování ZPL kódu pro tisk 2×1 palcových štítků
    \item Export BOM do CSV s timestampem
    \item Perzistence dat v JSON formátu
    \item Zobrazení detailních informací o součástce včetně scan historie
\end{enumerate}

\subsubsection{Nefunkční požadavky}

\begin{enumerate}[label=NFR\arabic*:]
    \item Rychlá odezva ($< 1$ sekundu pro zpracování QR kódu)
    \item Intuitivní GUI založené na PyQt6
    \item Offline funkčnost (TME API je volitelné)
    \item Multiplatformní kompatibilita (macOS, Windows, Linux)
    \item Automatické ukládání dat při každé změně
    \item Bezpečné uložení historie -- žádné ztráty dat
\end{enumerate}

\subsection{Aktéři systému}

\begin{description}
    \item[Uživatel] Technik nebo skladník provádějící skenování a správu součástek
    \item[Zebra Čtečka] Externí hardware -- čtečka QR kódů fungující jako USB klávesnice
    \item[TME API] Externí API služba pro získání dodatečných informací o součástkách
    \item[Zebra Tiskárna] Externí hardware -- tiskárna štítků podporující ZPL příkazy
\end{description}

\subsection{Rozsah projektu}

Tento projekt pokrývá \textbf{kompletní workflow} od skenování součástek po tisk štítků skladových míst. Systém je plně funkční a používaný v produkčním prostředí.

\newpage

\section{Datový slovník}

Datový slovník obsahuje detailní popis všech tříd, jejich atributů, metod a vztahů v systému. Systém je organizován do čtyř vrstev: UI Layer, Domain Model, Business Logic a External Integration.

\subsection{UI Layer -- Vrstva uživatelského rozhraní}

\subsubsection{BOMScannerMainWindow}

\textbf{Typ:} UI třída (hlavní okno WMS aplikace) \\
\textbf{Účel:} Hlavní řídící třída aplikace zodpovědná za UI a koordinaci všech operací

\paragraph{Atributy:}
\begin{itemize}
    \item \texttt{settings: QSettings} -- Perzistentní nastavení aplikace (Qt framework)
    \item \texttt{scanned\_codes: List[Dict]} -- Kolekce všech naskenovaných součástek
    \item \texttt{projects\_data: List[Dict]} -- Kolekce všech projektů
    \item \texttt{storage\_locations\_data: List[Dict]} -- Kolekce skladových míst
    \item \texttt{bom\_table: QTableWidget} -- Qt widget pro zobrazení BOM tabulky
    \item \texttt{projects\_table: QTableWidget} -- Qt widget pro zobrazení projektů
    \item \texttt{storage\_table: QTableWidget} -- Qt widget pro zobrazení skladových míst
    \item \texttt{tme\_api: TMEAPI} -- Instance TME API klienta
\end{itemize}

\paragraph{Metody:}
\begin{itemize}
    \item \texttt{init\_ui()} -- Inicializuje uživatelské rozhraní
    \item \texttt{on\_scan\_received(code: str)} -- Event handler pro příjem naskenovaného kódu
    \item \texttt{add\_code\_to\_list(code: str)} -- Přidá kód do seznamu
    \item \texttt{export\_to\_json()} -- Exportuje BOM do JSON formátu
    \item \texttt{export\_to\_csv()} -- Exportuje BOM do CSV formátu
    \item \texttt{save\_bom()} -- Uloží BOM do JSON souboru
    \item \texttt{load\_bom()} -- Načte BOM z JSON souboru
    \item \texttt{refresh\_projects\_table()} -- Aktualizuje zobrazení projektů
\end{itemize}

\subsubsection{PartDetailDialog}

\textbf{Typ:} UI třída (dialog) \\
\textbf{Účel:} Zobrazení detailních informací o součástce včetně historie skenování

\paragraph{Atributy:}
\begin{itemize}
    \item \texttt{parsed\_data: Dict} -- Parsovaná data součástky
    \item \texttt{history: List[str]} -- Seznam časových razítek skenování
    \item \texttt{raw\_code: str} -- Surový QR kód
    \item \texttt{part\_index: int} -- Index součástky v hlavní tabulce
    \item \texttt{history\_table: QTableWidget} -- Tabulka pro zobrazení historie
\end{itemize}

\paragraph{Metody:}
\begin{itemize}
    \item \texttt{manage\_projects()} -- Otevře dialog pro správu přiřazení k projektům
    \item \texttt{manage\_storage\_locations()} -- Otevře dialog pro správu skladových míst
    \item \texttt{delete\_selected\_history()} -- Smaže vybraný záznam z historie
    \item \texttt{clear\_history()} -- Vymaže celou historii skenování
\end{itemize}

\subsubsection{ZPLGeneratorTab}

\textbf{Typ:} UI třída (záložka) \\
\textbf{Účel:} Generování ZPL kódu pro tisk štítků skladových míst

\paragraph{Atributy:}
\begin{itemize}
    \item \texttt{location\_input: QLineEdit} -- Vstupní pole pro kód lokace
    \item \texttt{zpl\_output: QTextEdit} -- Výstupní pole s vygenerovaným ZPL
\end{itemize}

\paragraph{Metody:}
\begin{itemize}
    \item \texttt{generate\_zpl(location: str): str} -- Generuje ZPL kód pro lokaci
    \item \texttt{copy\_to\_clipboard()} -- Zkopíruje ZPL do schránky
    \item \texttt{save\_to\_file()} -- Uloží ZPL do .zpl souboru
\end{itemize}

\subsection{Domain Model -- Doménové entity}

\subsubsection{Part}

\textbf{Typ:} Entita (klíčová doménová třída) \\
\textbf{Účel:} Reprezentuje elektronickou součástku s jejími atributy a vztahy

\paragraph{Atributy:}
\begin{itemize}
    \item \texttt{pn: str} -- Part Number (primární identifikátor, např. ``R0805-100R'')
    \item \texttt{mpn: str} -- Manufacturer Part Number (např. ``RC0805FR-07100RL'')
    \item \texttt{manufacturer: str} -- Výrobce součástky (např. ``YAGEO'', ``Vishay'')
    \item \texttt{quantity: int} -- Aktuální množství kusů na skladě ($\geq 0$)
    \item \texttt{location: str} -- Kód skladového místa (např. ``A1'', ``SHELF-12'')
    \item \texttt{value: str} -- Hodnota součástky (např. ``100R'', ``10uF'', ``BC547'')
    \item \texttt{category: str} -- Kategorie součástky (např. ``Resistors'', ``Capacitors'')
    \item \texttt{coo: str} -- Country of Origin -- země původu (ISO kód, např. ``CN'', ``US'')
    \item \texttt{po: str} -- Purchase Order -- číslo nákupní objednávky
    \item \texttt{url: str} -- URL odkaz na datasheet nebo produktovou stránku
    \item \texttt{rohs: bool} -- RoHS compliance (True = splňuje, False = nesplňuje)
    \item \texttt{projects: List[str]} -- Seznam názvů projektů, ke kterým je součástka přiřazena
    \item \texttt{scan\_history: List[str]} -- Časová razítka všech skenování (ISO formát)
\end{itemize}

\paragraph{Metody:}
\begin{itemize}
    \item \texttt{add\_quantity(qty: int)} -- Přičte množství k existujícímu stavu
    \item \texttt{update\_location(location: str)} -- Aktualizuje skladové místo
    \item \texttt{add\_to\_project(project\_name: str)} -- Přiřadí součástku k projektu
    \item \texttt{remove\_from\_project(project\_name: str)} -- Odebere součástku z projektu
    \item \texttt{add\_scan\_timestamp()} -- Přidá aktuální čas do historie
    \item \texttt{get\_category(): str} -- Vrací kategorii součástky
    \item \texttt{to\_dict(): Dict} -- Serializuje objekt do slovníku
\end{itemize}

\paragraph{Kardinalita vazeb:}
\begin{itemize}
    \item Part $(0..*)$ --- $(0..*)$ Project (many-to-many)
    \item Part $(0..*)$ --- $(0..1)$ StorageLocation (many-to-one)
    \item Part $(1)$ --- $(1..*)$ ScanRecord (one-to-many, kompozice)
\end{itemize}

\subsubsection{Project}

\textbf{Typ:} Entita \\
\textbf{Účel:} Reprezentuje projekt, ke kterému lze přiřazovat součástky

\paragraph{Atributy:}
\begin{itemize}
    \item \texttt{name: str} -- Název projektu (jedinečný identifikátor)
    \item \texttt{description: str} -- Popis projektu
    \item \texttt{created\_date: str} -- Datum vytvoření (ISO formát YYYY-MM-DD)
    \item \texttt{parts: List[str]} -- Seznam Part Numbers přiřazených součástek
\end{itemize}

\paragraph{Metody:}
\begin{itemize}
    \item \texttt{add\_part(pn: str)} -- Přidá součástku do projektu
    \item \texttt{remove\_part(pn: str)} -- Odebere součástku z projektu
    \item \texttt{get\_parts\_count(): int} -- Vrací počet součástek v projektu
    \item \texttt{to\_dict(): Dict} -- Serializace
    \item \texttt{from\_dict(data: Dict): Project} -- Deserializace (class method)
\end{itemize}

\subsubsection{StorageLocation}

\textbf{Typ:} Entita \\
\textbf{Účel:} Reprezentuje fyzické skladové místo (police, box, zásuvka)

\paragraph{Atributy:}
\begin{itemize}
    \item \texttt{code: str} -- Kód skladového místa (např. ``A1'', ``B23'', jedinečný)
    \item \texttt{description: str} -- Popis umístění (např. ``Levá police, horní řada'')
    \item \texttt{created\_date: str} -- Datum vytvoření (ISO formát)
    \item \texttt{parts: List[str]} -- Seznam Part Numbers uložených součástek
\end{itemize}

\paragraph{Metody:}
\begin{itemize}
    \item \texttt{assign\_part(pn: str)} -- Přiřadí součástku na toto místo
    \item \texttt{remove\_part(pn: str)} -- Odebere součástku z místa
    \item \texttt{is\_empty(): bool} -- Vrací True pokud žádné součástky
    \item \texttt{to\_dict(): Dict} -- Serializace
\end{itemize}

\subsubsection{ScanRecord}

\textbf{Typ:} Entita \\
\textbf{Účel:} Záznam o jednom skenování QR kódu

\paragraph{Atributy:}
\begin{itemize}
    \item \texttt{code: str} -- Surový text z QR kódu
    \item \texttt{timestamp: str} -- Čas skenování (ISO formát YYYY-MM-DD HH:MM:SS)
    \item \texttt{parsed\_data: Dict} -- Parsovaná data ze skenování
    \item \texttt{length: int} -- Délka kódu v znacích
\end{itemize}

\subsection{Business Logic -- Manažerské třídy}

\subsubsection{BOMManager}

\textbf{Typ:} Manager (business logika) \\
\textbf{Účel:} Centrální správce všech součástek v BOM

\paragraph{Atributy:}
\begin{itemize}
    \item \texttt{parts: Dict[str, Part]} -- Slovník součástek indexovaný podle PN (kvalifikovaná asociace)
    \item \texttt{filename: str} -- Cesta k JSON souboru s BOM daty
\end{itemize}

\paragraph{Metody:}
\begin{itemize}
    \item \texttt{add\_or\_update\_part(parsed\_data: Dict): Part} -- Přidá novou nebo aktualizuje existující
    \item \texttt{get\_part(pn: str): Part} -- Získá součástku podle PN
    \item \texttt{remove\_part(pn: str)} -- Odstraní součástku z BOM
    \item \texttt{get\_all\_parts(): List[Part]} -- Vrací všechny součástky
    \item \texttt{save(): bool} -- Uloží BOM do JSON
    \item \texttt{load(): bool} -- Načte BOM z JSON
    \item \texttt{export\_to\_csv(filename: str)} -- Export do CSV
    \item \texttt{get\_parts\_by\_project(project: str): List[Part]} -- Filtr podle projektu
    \item \texttt{get\_parts\_by\_location(location: str): List[Part]} -- Filtr podle lokace
\end{itemize}

\paragraph{Vztah k Part:}
\begin{itemize}
    \item Agregace: BOMManager $(1)$ $\circ$-- $(0..*)$ Part
    \item Kvalifikátor: PN (Part Number) pro rychlý $O(1)$ přístup
\end{itemize}

\subsubsection{QRParser}

\textbf{Typ:} Utility (statická třída) \\
\textbf{Účel:} Parsování QR kódů do strukturovaných dat

\paragraph{Metody:}
\begin{itemize}
    \item \texttt{parse\_qr\_code(code: str): Dict} -- Hlavní parsovací metoda
    \item \texttt{extract\_field(code: str, field: str): str} -- Extrahuje konkrétní pole
    \item \texttt{parse\_quantity(qty\_str: str): int} -- Parsuje množství
    \item \texttt{validate\_format(code: str): bool} -- Validuje formát QR kódu
\end{itemize}

\paragraph{Formát vstupu:}
\begin{lstlisting}[language=Python]
PN=hodnota,MPN=hodnota,QTY=hodnota,...
\end{lstlisting}

\subsection{External Integration -- Externí integrace}

\subsubsection{TMEAPI}

\textbf{Typ:} External API Client \\
\textbf{Účel:} Integrace s TME (Transfer Multisort Elektronik) API

\paragraph{Atributy:}
\begin{itemize}
    \item \texttt{token: str} -- Private token (50 znaků, autentizace)
    \item \texttt{app\_secret: str} -- Application secret (20 znaků, pro signaturu)
    \item \texttt{base\_url: str} -- ``https://api.tme.eu''
    \item \texttt{default\_country: str} -- ``CZ'' (Czech Republic)
\end{itemize}

\paragraph{Metody:}
\begin{itemize}
    \item \texttt{search\_products(symbol: str): Dict} -- Vyhledá součástku podle symbolu/MPN
    \item \texttt{get\_product\_details(symbol: str): Dict} -- Získá detailní informace
    \item \texttt{\_generate\_signature(method, uri, params): str} -- HMAC-SHA1 signatura
    \item \texttt{\_make\_request(action: str, params: Dict): Dict} -- HTTP POST request
\end{itemize}

\subsection{Invarianty systému}

\begin{enumerate}
    \item $\texttt{Part.quantity} \geq 0$ -- Množství nemůže být záporné
    \item \texttt{Part.pn} je jedinečný v rámci BOMManager
    \item \texttt{Project.name} je jedinečný v rámci ProjectManager
    \item \texttt{StorageLocation.code} je jedinečný v rámci StorageManager
    \item Součástka může být přiřazena maximálně k jednomu skladovému místu
    \item Součástka může být přiřazena k více projektům
    \item Všechny timestamp jsou v ISO formátu ``YYYY-MM-DD HH:MM:SS''
\end{enumerate}

\newpage

\section{Objektový model}

Objektový model systému je organizován do čtyř vrstev s jasnou separací zodpovědností. Celkem obsahuje 19 tříd propojených různými typy vazeb.

\subsection{Class Diagram}

\begin{figure}[H]
    \centering
    \includegraphics[width=\textwidth,height=0.9\textheight,keepaspectratio]{BOM Manager - Class Diagram.png}
    \caption{Class Diagram systému BOM Manager}
    \label{fig:class-diagram}
\end{figure}

\subsection{Architektura systému}

Systém využívá vrstvenou architekturu:

\begin{enumerate}
    \item \textbf{UI Layer} -- Uživatelské rozhraní (PyQt6 komponenty)
    \item \textbf{Business Logic} -- Manažerské třídy a business pravidla
    \item \textbf{Domain Model} -- Doménové entity (Part, Project, StorageLocation)
    \item \textbf{External Integration} -- API klienti a drivery
\end{enumerate}

\subsection{Klíčové vlastnosti objektového modelu}

\subsubsection{Agregace a Kompozice}

\paragraph{Agregace ($\circ$--):}
Části mohou existovat nezávisle na kontejneru.
\begin{itemize}
    \item BOMManager $(1)$ $\circ$-- $(0..*)$ Part
    \item ProjectManager $(1)$ $\circ$-- $(0..*)$ Project
    \item StorageManager $(1)$ $\circ$-- $(0..*)$ StorageLocation
\end{itemize}

\paragraph{Kompozice ($\bullet$--):}
Části nemohou existovat bez vlastníka.
\begin{itemize}
    \item Part $(1)$ $\bullet$-- $(1..*)$ ScanRecord
    \item BOMScannerMainWindow $(1)$ $\bullet$-- $(1)$ ZPLGeneratorTab
\end{itemize}

\subsubsection{Kvalifikované vazby}

Kvalifikované asociace umožňují rychlý $O(1)$ přístup k objektům pomocí klíče:

\begin{enumerate}
    \item \textbf{BOMManager $\rightarrow$ Part} kvalifikováno pomocí \texttt{pn: str}
    \begin{lstlisting}[language=Python]
BOMManager[pn: str] -> Part
    \end{lstlisting}
    
    \item \textbf{ProjectManager $\rightarrow$ Project} kvalifikováno pomocí \texttt{name: str}
    
    \item \textbf{StorageManager $\rightarrow$ StorageLocation} kvalifikováno pomocí \texttt{code: str}
\end{enumerate}

Implementace pomocí Python slovníků:
\begin{lstlisting}[language=Python]
class BOMManager:
    def __init__(self):
        self.parts: Dict[str, Part] = {}
    
    def get_part(self, pn: str) -> Part:
        return self.parts.get(pn)  # O(1) pristup
\end{lstlisting}

\subsubsection{Kardinalita vazeb}

\begin{itemize}
    \item Part $(0..*)$ --- $(0..*)$ Project \\ 
          \textit{Many-to-Many: součástka může být ve více projektech, projekt obsahuje více součástek}
    
    \item Part $(0..*)$ --- $(0..1)$ StorageLocation \\
          \textit{Many-to-One: součástka může být max. na jednom místě, místo může obsahovat více součástek}
    
    \item Part $(1)$ $\bullet$-- $(1..*)$ ScanRecord \\
          \textit{One-to-Many kompozice: každá součástka má alespoň jeden scan record}
\end{itemize}

\subsubsection{Atributy spojení}

Pro M:N vztah Part --- Project obě strany udržují seznamy:
\begin{itemize}
    \item \texttt{Part.projects: List[str]} -- seznam názvů projektů
    \item \texttt{Project.parts: List[str]} -- seznam Part Numbers
\end{itemize}

Tato implementace umožňuje navigaci v obou směrech bez nutnosti samostatné asociační třídy.

\subsection{Design patterns}

\subsubsection{Manager Pattern}
Třídy BOMManager, ProjectManager, StorageManager zapouzdřují logiku správy kolekcí a poskytují jednotné rozhraní.

\subsubsection{Static Utility Classes}
QRParser, ZPLGenerator, CategoryMapper poskytují statické metody bez nutnosti vytváření instancí.

\subsubsection{Dependency Injection}
TMEAPI je injektován do BOMScannerMainWindow, umožňuje snadné testování a výměnu implementace.

\newpage

\section{Stavový model}

Stavový model popisuje dynamické chování klíčových tříd systému. Implementovali jsme dva komplexní stavové diagramy pro třídy Part a BOMScanner.

\subsection{State Machine Diagram -- Part}

\begin{figure}[H]
    \centering
    \includegraphics[width=\textwidth,height=0.85\textheight,keepaspectratio]{Part - State Machine Diagram.png}
    \caption{Stavový diagram třídy Part}
    \label{fig:state-part}
\end{figure}

\subsubsection{Popis stavů součástky}

\paragraph{New} -- Inicializační stav
\begin{itemize}
    \item Součástka právě naskenována
    \item Vnořené stavy: Parsing $\rightarrow$ Validated / Invalid
    \item Entry action: \texttt{parse\_qr\_code()}
    \item Přechod: po validaci $\rightarrow$ InBOM
\end{itemize}

\paragraph{InBOM} -- Složený stav s vnořenými substates
Tento složený stav reprezentuje součástku aktivně vedenou v BOM s různými úrovněmi alokace:

\begin{description}
    \item[Unallocated] -- Bez přiřazení lokace i projektu
    \item[PartiallyAllocated] -- Má lokaci NEBO projekty
        \begin{itemize}
            \item HasLocation -- Přiřazeno skladové místo
            \item HasLocationAndProjects -- Má lokaci i projekty
        \end{itemize}
    \item[FullyAllocated] -- Kompletně alokováno (lokace + projekty + štítek)
\end{description}

\paragraph{QuantityChanged} -- Přechodný stav
\begin{itemize}
    \item Entry actions: \texttt{update\_quantity()}, \texttt{add\_scan\_timestamp()}
    \item Do activity: \texttt{refresh\_display()}
    \item Automatický přechod zpět do InBOM po aktualizaci
\end{itemize}

\paragraph{LowStock} -- Varování
\begin{itemize}
    \item Guard: $[\texttt{quantity} < \texttt{threshold}]$
    \item Entry action: \texttt{show\_warning()}
    \item Přechod zpět do InBOM při doplnění zásob
\end{itemize}

\paragraph{OutOfStock} -- Vyprodáno
\begin{itemize}
    \item Guard: $[\texttt{quantity} == 0]$
    \item Entry action: \texttt{mark\_as\_empty()}
    \item Nelze přiřadit k projektu v tomto stavu
\end{itemize}

\paragraph{Archived} -- Finální stav
\begin{itemize}
    \item Entry action: \texttt{remove\_from\_active\_bom()}
    \item Historie zachována pro audit
\end{itemize}

\subsubsection{Klíčové přechody}

\begin{itemize}
    \item \texttt{scan\_again() / add\_quantity()} -- Re-skenování existující součástky
    \item \texttt{assign\_location()} -- Přiřazení skladového místa
    \item \texttt{assign\_to\_project()} -- Přiřazení k projektu
    \item \texttt{print\_label()} -- Tisk štítku, přechod do FullyAllocated
    \item \texttt{delete()} -- Smazání součástky
\end{itemize}

\subsection{State Machine Diagram -- BOMScanner}

\begin{figure}[H]
    \centering
    \includegraphics[width=\textwidth,height=0.85\textheight,keepaspectratio]{BOMScanner - State Machine Diagram.png}
    \caption{Stavový diagram třídy BOMScanner (hlavní aplikace)}
    \label{fig:state-scanner}
\end{figure}

\subsubsection{Popis stavů aplikace}

\paragraph{Initializing} -- Inicializace aplikace
Složený stav s posloupností kroků:
\begin{enumerate}
    \item LoadingSettings -- \texttt{entry / load\_qsettings()}
    \item LoadingData -- \texttt{entry / load\_bom\_from\_json()}, \texttt{load\_projects()}, \texttt{load\_storage\_locations()}
    \item InitializingUI -- \texttt{entry / create\_widgets()}, \texttt{setup\_layouts()}
    \item ConnectingAPI -- \texttt{entry / initialize\_tme\_api()}
\end{enumerate}

Guard: $[\text{success}]$ -- pokud API nepřipojeno, pokračuje bez integrace.

\paragraph{Ready} -- Složený stav připravenosti
Hlavní operační stav aplikace s vnořenými substates:

\begin{description}
    \item[Idle] -- Čeká na vstup uživatele
        \begin{itemize}
            \item WaitingForScan -- Textové pole aktivní
            \item Exit action: \texttt{clear\_input\_field()}
        \end{itemize}
    
    \item[Processing] -- Zpracování QR kódu
        \begin{itemize}
            \item ParsingCode
            \item CheckingExistence
            \item \textbf{Fork:} UpdatingExisting $|$ CreatingNew
            \item \textbf{Join:} SavingData
        \end{itemize}
    
    \item[ManagingData] -- Uživatelské akce
        \begin{itemize}
            \item \textbf{Choice pseudostate} pro směrování:
            \item ViewingDetails
            \item EditingProjects
            \item AllocatingStorage
            \item PrintingLabels
            \item ExportingData
        \end{itemize}
\end{description}

\paragraph{Saving} -- Ukládání před zavřením
\begin{itemize}
    \item Entry actions: \texttt{save\_bom()}, \texttt{save\_projects()}, \texttt{save\_storage\_locations()}, \texttt{save\_settings()}
    \item Automatické uložení všech dat
\end{itemize}

\subsubsection{Fork/Join konstrukce}

Processing stav obsahuje fork/join pro paralelní zpracování:

\begin{verbatim}
CheckingExistence
    | fork
    +-> UpdatingExisting (soucástka existuje)
    +-> CreatingNew (nová soucástka)
        +-> QueryingTME
        +-> UsingQRDataOnly
    | join
SavingData
\end{verbatim}

\subsubsection{Choice pseudostate}

ManagingData používá choice pro směrování podle typu uživatelské akce:

\begin{itemize}
    \item $[\text{view\_part}]$ $\rightarrow$ ViewingDetails
    \item $[\text{manage\_projects}]$ $\rightarrow$ EditingProjects
    \item $[\text{allocate\_storage}]$ $\rightarrow$ AllocatingStorage
    \item $[\text{print\_label}]$ $\rightarrow$ PrintingLabels
    \item $[\text{export}]$ $\rightarrow$ ExportingData
\end{itemize}

\subsubsection{Entry/Exit/Do akce}

\begin{itemize}
    \item \textbf{Entry:} Akce provedená při vstupu do stavu (např. \texttt{entry / load\_qsettings()})
    \item \textbf{Exit:} Akce provedená při opuštění stavu (např. \texttt{exit / clear\_input\_field()})
    \item \textbf{Do:} Aktivita probíhající po celou dobu stavu (např. \texttt{do / search\_in\_bom()})
\end{itemize}

\subsection{Propojení objektového a stavového modelu}

\begin{itemize}
    \item Stavy odpovídají hodnotám atributů (např. \texttt{quantity == 0} $\rightarrow$ OutOfStock)
    \item Přechody odpovídají volání metod (např. \texttt{add\_quantity()} $\rightarrow$ QuantityChanged)
    \item Guards používají invarianty systému (např. $[\texttt{quantity} \geq 0]$)
    \item Entry/exit akce volají metody tříd definované v objektovém modelu
\end{itemize}

\newpage

\section{Model interakcí}

Model interakcí zachycuje komunikaci mezi objekty, scénáře použití a workflow systému z různých pohledů.

\subsection{Use Case Diagram}

\begin{figure}[H]
    \centering
    \includegraphics[width=\textwidth,height=0.85\textheight,keepaspectratio]{BOM Manager - Use Case Diagram.png}
    \caption{Use Case Diagram systému BOM Manager}
    \label{fig:usecase}
\end{figure}

\subsubsection{Přehled Use Cases}

Systém obsahuje celkem \textbf{31 use cases} organizovaných do 5 balíčků:

\paragraph{Skenování a Správa Součástek (6 UC):}
\begin{itemize}
    \item UC1: Naskenovat QR kód
    \item UC2: Parsovat data z QR kódu (<<include>> z UC1)
    \item UC3: Zobrazit detail součástky
    \item UC4: Upravit množství
    \item UC5: Smazat součástku
    \item UC15: Načíst QR z obrázku
\end{itemize}

\paragraph{Správa Projektů (4 UC):}
\begin{itemize}
    \item UC6: Vytvořit projekt
    \item UC7: Přiřadit součástku k projektu (<<extend>> z UC3)
    \item UC8: Zobrazit součástky projektu
    \item UC9: Upravit projekt
\end{itemize}

\paragraph{Správa Skladových Míst (3 UC):}
\begin{itemize}
    \item UC10: Vytvořit skladové místo
    \item UC11: Přiřadit součástku na místo (<<extend>> z UC3)
    \item UC12: Vytisknout štítek skladového místa
\end{itemize}

\paragraph{Export a Import (3 UC):}
\begin{itemize}
    \item UC13: Exportovat BOM do CSV
    \item UC14: Exportovat do JSON
    \item UC16: Importovat BOM
\end{itemize}

\paragraph{TME Integrace (3 UC):}
\begin{itemize}
    \item UC17: Vyhledat součástku v TME
    \item UC18: Získat informace o součástce (<<include>> z UC17)
    \item UC19: Aktualizovat cenu (<<extend>> z UC18)
\end{itemize}

\subsubsection{Vztahy mezi Use Cases}

\begin{description}
    \item[<<include>>] Povinná závislost, vždy se provede
    \begin{itemize}
        \item UC1 $\rightarrow$ UC2: Skenování vždy zahrnuje parsování
        \item UC17 $\rightarrow$ UC18: Vyhledání zahrnuje získání informací
    \end{itemize}
    
    \item[<<extend>>] Volitelné rozšíření
    \begin{itemize}
        \item UC3 $\rightarrow$ UC7: Detail součástky může rozšířit o přiřazení k projektu
        \item UC3 $\rightarrow$ UC11: Detail součástky může rozšířit o přiřazení lokace
        \item UC2 $\rightarrow$ UC18: Parsování může rozšířit o TME data
    \end{itemize}
\end{description}

\subsection{Scénáře Use Cases}

\subsubsection{UC1: Naskenovat QR kód}

\paragraph{Hlavní scénář (Success):}
\begin{enumerate}
    \item Uživatel spustí aplikaci BOM Manager
    \item Systém zobrazí hlavní okno s aktivním vstupním polem
    \item Uživatel klikne do vstupního pole pro skenování
    \item Uživatel naskenuje QR kód pomocí Zebra čtečky
    \item Čtečka odošle data do vstupního pole a automaticky stiskne Enter
    \item Systém parsuje QR kód a extrahuje data (PN, MPN, QTY, MFR, atd.)
    \item Systém kontroluje, zda součástka již existuje v BOM
    \item Součástka neexistuje -- systém vytvoří nový záznam
    \item Systém dotazuje TME API pro získání dodatečných informací
    \item Systém přidá součástku do BOM tabulky
    \item Systém automaticky uloží BOM do JSON souboru
    \item Systém vymaže vstupní pole a zobrazí potvrzení
    \item \textbf{Use case končí úspěchem}
\end{enumerate}

\paragraph{Alternativní scénář A1: Součástka již existuje}
\begin{itemize}
    \item Začíná v kroku 7
    \item A1.1: Systém najde existující součástku
    \item A1.2: Systém přičte naskenované množství k existujícímu
    \item A1.3: Systém přidá timestamp do scan historie
    \item A1.4: Systém aktualizuje zobrazení v tabulce
    \item A1.5: Pokračuje krokem 11
\end{itemize}

\paragraph{Alternativní scénář A2: TME API nedostupné}
\begin{itemize}
    \item Začíná v kroku 9
    \item A2.1: TME API neodpovídá nebo vrací chybu
    \item A2.2: Systém použije pouze data z QR kódu
    \item A2.3: Systém nastaví kategorii na ``Unknown''
    \item A2.4: Pokračuje krokem 10
\end{itemize}

\paragraph{Výjimečný scénář E1: Nevalidní QR kód}
\begin{itemize}
    \item Začíná v kroku 6
    \item E1.1: Parsování selže, data nejsou ve správném formátu
    \item E1.2: Systém zobrazí chybovou hlášku
    \item E1.3: Systém vymaže vstupní pole
    \item E1.4: \textbf{Use case končí neúspěchem}
\end{itemize}

\paragraph{Preconditions:}
\begin{itemize}
    \item Aplikace je spuštěna
    \item Zebra čtečka je připojena k počítači
    \item Vstupní pole je aktivní (focused)
\end{itemize}

\paragraph{Postconditions:}
\begin{itemize}
    \item \textbf{Success:} Součástka je přidána/aktualizována v BOM a uložena do JSON
    \item \textbf{Failure:} Vstupní pole je vymazáno, chyba zobrazena uživateli
\end{itemize}

\subsubsection{UC11: Přiřadit součástku na skladové místo}

\paragraph{Hlavní scénář:}
\begin{enumerate}
    \item Uživatel otevře záložku ``Allocating Storage Locations''
    \item Systém zobrazí seznam všech součástek a skladových míst
    \item Uživatel vybere součástku ze seznamu
    \item Uživatel vybere skladové místo z dropdown menu
    \item Uživatel klikne ``Assign Location''
    \item Systém ověří, že součástka ještě nemá přiřazené místo
    \item Systém přiřadí součástku na vybrané skladové místo
    \item Systém aktualizuje sloupec ``Storage Location'' v BOM tabulce
    \item Systém uloží změny do storage\_locations.json a BOM\_current.csv
    \item Systém zobrazí potvrzovací zprávu
    \item Systém aktualizuje počet součástek na skladovém místě
    \item \textbf{Use case končí úspěchem}
\end{enumerate}

\paragraph{Alternativní scénář A1: Součástka již má místo}
\begin{itemize}
    \item Začíná v kroku 6
    \item A1.1: Systém detekuje existující přiřazení
    \item A1.2: Systém zobrazí dialog s varováním
    \item A1.3: Uživatel potvrdí přepsání nebo zruší
    \item A1.4: Pokud potvrzeno, pokračuje krokem 7
    \item A1.5: Pokud zrušeno, use case končí
\end{itemize}

\subsubsection{UC12: Vytisknout štítek skladového místa}

\paragraph{Hlavní scénář:}
\begin{enumerate}
    \item Uživatel otevře záložku ``Print Labels''
    \item Systém zobrazí formulář pro generování štítků
    \item Uživatel zadá kód skladového místa (např. ``A1'', ``B23'')
    \item Systém automaticky generuje ZPL kód při psaní
    \item Systém zobrazí náhled ZPL kódu v textovém poli
    \item Uživatel klikne ``Copy to Clipboard''
    \item Systém zkopíruje ZPL kód do schránky
    \item Systém zobrazí potvrzení ``ZPL copied to clipboard''
    \item Uživatel otevře Zebra Setup Utilities
    \item Uživatel vloží ZPL kód do aplikace tiskárny
    \item Uživatel odešle příkaz k tisku
    \item Zebra tiskárna vytiskne štítek 2×1 palec s čárovým kódem
    \item \textbf{Use case končí úspěchem}
\end{enumerate}

\paragraph{ZPL formát:}
\begin{lstlisting}
^XA
^FO50,20^BY2^BCN,100,Y,N,N^FDA1^FS
^FO50,140^A0N,30,30^FDStorage: A1^FS
^XZ
\end{lstlisting}

\subsection{Sekvenční diagram}

\begin{figure}[H]
    \centering
    \includegraphics[width=\textwidth,height=0.85\textheight,keepaspectratio]{BOM Manager - Sekvenční Diagram (Skenování a Přiřazení Součástky).png}
    \caption{Sekvenční diagram workflow skenování a přiřazení}
    \label{fig:sequence}
\end{figure}

\subsubsection{Popis sekvence}

\paragraph{Inicializace aplikace:}
\begin{enumerate}
    \item Uživatel $\rightarrow$ GUI: \texttt{Spustit aplikaci}
    \item GUI $\rightarrow$ DB: \texttt{Načíst existující BOM}
    \item GUI $\rightarrow$ DB: \texttt{Načíst projekty}
    \item GUI $\rightarrow$ DB: \texttt{Načíst skladová místa}
    \item GUI $\rightarrow$ Uživatel: \texttt{Zobrazit hlavní okno}
\end{enumerate}

\paragraph{Skenování QR kódu:}
\begin{enumerate}
    \item Scanner $\rightarrow$ GUI: \texttt{Odeslat data (Enter)}
    \item GUI $\rightarrow$ Parser: \texttt{Parsovat QR data}
    \item Parser $\rightarrow$ GUI: \texttt{Parsed data}
    \item \textbf{Alt:} Součástka existuje / neexistuje
    \item Pokud neexistuje: GUI $\rightarrow$ TME: \texttt{Vyhledat součástku (MPN)}
    \item TME $\rightarrow$ GUI: \texttt{Informace o součástce}
    \item GUI $\rightarrow$ BOM: \texttt{Přidat novou součástku}
    \item GUI $\rightarrow$ DB: \texttt{Uložit aktualizovaný BOM}
\end{enumerate}

\paragraph{Přiřazení skladového místa:}
\begin{enumerate}
    \item Uživatel $\rightarrow$ GUI: \texttt{Otevřít detail součástky}
    \item Uživatel $\rightarrow$ GUI: \texttt{Kliknout "Manage Storage Locations"}
    \item GUI $\rightarrow$ Storage: \texttt{Přiřadit součástku na místo}
    \item GUI $\rightarrow$ DB: \texttt{Uložit změnu}
\end{enumerate}

\subsubsection{Klíčové vlastnosti}

\begin{itemize}
    \item \textbf{Životnost objektů:} Aktivační bloky (tenké obdélníky) ukazují dobu zpracování
    \item \textbf{Synchronní zprávy:} Plné šipky (např. \texttt{parse\_qr\_code()})
    \item \textbf{Návratové hodnoty:} Přerušované šipky
    \item \textbf{Alt fragment:} Podmíněné chování (existuje/neexistuje)
\end{itemize}

\subsection{Diagram aktivit}

\begin{figure}[H]
    \centering
    \includegraphics[width=\textwidth,height=0.85\textheight,keepaspectratio]{BOM Manager - Diagram Aktivit (Kompletní Workflow).png}
    \caption{Diagram aktivit kompletního workflow}
    \label{fig:activity}
\end{figure}

\subsubsection{Popis workflow}

\paragraph{Start $\rightarrow$ Inicializace:}
\begin{itemize}
    \item Načtení dat z JSON
    \item Inicializace TME API
    \item Zobrazení GUI
\end{itemize}

\paragraph{Cyklus skenování (repeat loop):}
\begin{enumerate}
    \item Naskenovat QR kód (Zebra Čtečka)
    \item Parsovat data
    \item \textbf{Decision:} Součástka existuje?
    \begin{itemize}
        \item ANO: Přičíst množství + timestamp
        \item NE: \textbf{Fork} $\rightarrow$ Vytvořit záznam $|$ Vyhledat v TME $\rightarrow$ \textbf{Join}
    \end{itemize}
    \item Aktualizovat UI
    \item Auto-save
    \item \textbf{Decision:} Další součástka?
\end{enumerate}

\paragraph{Volitelné kroky:}
\begin{itemize}
    \item \textbf{Decision:} Přiřadit skladová místa? (repeat loop)
    \item \textbf{Decision:} Tisknout štítky? (repeat loop)
    \begin{itemize}
        \item \textbf{Choice:} clipboard / soubor
    \end{itemize}
    \item \textbf{Decision:} Přiřadit k projektům? (repeat loop)
\end{itemize}

\paragraph{Export (volitelný):}
\begin{itemize}
    \item \textbf{Decision:} Formát? CSV / JSON
    \item Vytvořit soubor
\end{itemize}

\paragraph{Ukončení:}
\begin{itemize}
    \item Auto-save všech dat
    \item Uložit settings
    \item Stop
\end{itemize}

\subsubsection{Použité elementy}

\begin{itemize}
    \item \textbf{Swimlanes:} Uživatel, Systém, Zebra Čtečka, Zebra Tiskárna
    \item \textbf{Decision nodes:} Diamanty pro podmínky
    \item \textbf{Fork/Join:} Paralelní zpracování (vytvoření + TME dotaz)
    \item \textbf{Loop nodes:} Repeat cykly pro opakované akce
    \item \textbf{Choice nodes:} Výběr cesty (clipboard vs. soubor)
    \item \textbf{Note elements:} Vysvětlivky k aktivitám
\end{itemize}

\newpage

\section{Závěr}

\subsection{Zhodnocení projektu}

Projekt WMS pro správu elektronických součástek úspěšně demonstruje aplikaci objektově orientované metodologie UML na reálný problém správy součástek ve výrobním prostředí.

\subsubsection{Splnění požadavků zadání}

\begin{enumerate}
    \item \textbf{Formulace problému} \\
    Detailně popsán problém správy BOM, cíle systému, funkční i nefunkční požadavky a aktéři.
    
    \item \textbf{Datový slovník} \\
    Kompletní popis 19 tříd včetně atributů, metod, kardinalit a invariantů systému.
    
    \item \textbf{Objektový model} \\
    Class diagram s využitím:
    \begin{itemize}
        \item Agregace (\texttt{BOMManager} $\circ$-- \texttt{Part})
        \item Kompozice (\texttt{Part} $\bullet$-- \texttt{ScanRecord})
        \item Kvalifikované vazby (\texttt{Dict[pn, Part]})
        \item Kardinalita vazeb (M:N, M:1, 1:M)
        \item Stereotypy (<<Entity>>, <<UI>>, <<Manager>>, <<API>>)
    \end{itemize}
    
    \item \textbf{Stavový model} \\
    Dva komplexní State Machine diagramy:
    \begin{itemize}
        \item \textbf{Part:} 6 hlavních stavů, 3 složené stavy, entry/exit/do akce
        \item \textbf{BOMScanner:} Fork/join konstrukce, choice pseudostate, guards
    \end{itemize}
    
    \item \textbf{Model interakcí} \\
    \begin{itemize}
        \item Use Case diagram: 31 use cases, include/extend vztahy
        \item Scénáře: Hlavní, alternativní a výjimečné scénáře
        \item Sekvenční diagram: Interakce mezi 9 objekty
        \item Diagram aktivit: Kompletní workflow s decision/fork/loop nodes
    \end{itemize}
    
    \item \textbf{Závěr} \\
    Zhodnocení projektu, propojení modelů, výhody OOP, využití AI
\end{enumerate}

\subsection{Propojení modelů}

Všechny vytvořené modely jsou navzájem konzistentní a vzájemně se doplňují:

\paragraph{Objektový $\leftrightarrow$ Stavový}
\begin{itemize}
    \item Stavy odpovídají hodnotám atributů (\texttt{quantity == 0} $\rightarrow$ OutOfStock)
    \item Přechody odpovídají metodám tříd (\texttt{add\_quantity()} $\rightarrow$ QuantityChanged)
    \item Guards používají invarianty systému
\end{itemize}

\paragraph{Objektový $\leftrightarrow$ Use Case}
\begin{itemize}
    \item Každý use case je realizován metodami tříd
    \item UC1 (Naskenovat) $\rightarrow$ \texttt{BOMScannerMainWindow.on\_scan\_received()}
    \item UC12 (Tisk štítku) $\rightarrow$ \texttt{ZPLGenerator.generate\_label()}
\end{itemize}

\paragraph{Stavový $\leftrightarrow$ Sekvenční}
\begin{itemize}
    \item Stavy v State Diagram odpovídají fázím v Sequence Diagram
    \item Part.New $\rightarrow$ Sekvenční: Parser.parse()
    \item BOMScanner.Processing $\rightarrow$ Celá sekvence zpracování QR
\end{itemize}

\paragraph{Use Case $\leftrightarrow$ Sekvenční $\leftrightarrow$ Aktivit}
\begin{itemize}
    \item UC1 (Naskenovat) je detailně rozpracován v obou diagramech
    \item Sequence Diagram: Interakce mezi objekty
    \item Activity Diagram: Tok aktivit s rozhodovacími body
\end{itemize}

\subsection{Výhody objektového přístupu}

Objektově orientovaný návrh přinesl systému následující výhody:

\begin{enumerate}
    \item \textbf{Modularita} \\
    Třídy jsou nezávislé, změna jedné neovlivní ostatní. Například změna implementace QRParser neovlivní GUI.
    
    \item \textbf{Znovupoužitelnost} \\
    Manager třídy (BOMManager, ProjectManager) lze použít i v jiných projektech. TME API klient je zcela nezávislý.
    
    \item \textbf{Rozšiřitelnost} \\
    Snadné přidání nových funkcí díky jasným rozhraním. Například přidání nového API pouze vyžaduje vytvoření nové třídy implementující stejné rozhraní.
    
    \item \textbf{Údržba} \\
    Jasná struktura usnadňuje hledání a opravu chyb. Každá vrstva má jasně definovanou zodpovědnost.
    
    \item \textbf{Testovatelnost} \\
    Každá třída a metoda je testovatelná samostatně díky loose coupling.
    
    \item \textbf{Separace zodpovědností} \\
    UI, business logika, doména a integrace jsou oddělené, což umožňuje paralelní vývoj.
\end{enumerate}

\subsection{Využití umělé inteligence}

Při realizaci tohoto projektu byla aktivně využita \textbf{umělá inteligence (GitHub Copilot)} v následujících oblastech:

\subsubsection{Analýza a návrh}

\begin{itemize}
    \item \textbf{Generování Class Diagramu:} AI asistovala při identifikaci vhodné struktury tříd a jejich vztahů
    \item \textbf{Návrh stavových diagramů:} Pomoc při identifikaci klíčových stavů a přechodů
    \item \textbf{Identifikace use cases:} AI navrhla kompletní seznam 31 use cases pokrývajících všechny funkce systému
\end{itemize}

\subsubsection{Dokumentace}

\begin{itemize}
    \item \textbf{Datový slovník:} AI vygenerovala strukturovaný popis všech 19 tříd včetně atributů a metod
    \item \textbf{Scénáře use cases:} Kompletní scénáře včetně alternativních a výjimečných cest
    \item \textbf{LaTeX dokument:} AI asistovala při vytváření tohoto PDF dokumentu včetně formátování
\end{itemize}

\subsubsection{Generování diagramů}

\begin{itemize}
    \item \textbf{PlantUML kód:} Všechny UML diagramy byly vygenerovány pomocí AI v PlantUML syntaxi
    \item \textbf{Komplexní diagramy:} AI zvládla vytvořit složité konstrukce jako fork/join, choice, composite states
    \item \textbf{Konzistence:} AI zajistila konzistenci mezi různými diagramy (stejné názvy tříd, metod, atributů)
\end{itemize}

\subsubsection{Výhody použití AI}

\begin{enumerate}
    \item \textbf{Rychlost:} Kompletní projekt včetně 6 UML diagramů vytvořen za několik hodin
    \item \textbf{Kvalita:} AI navrhla profesionální strukturu odpovídající best practices
    \item \textbf{Konzistence:} Automatické zajištění konzistence mezi modely
    \item \textbf{Dokumentace:} Komplexní dokumentace vygenerována okamžitě
    \item \textbf{Iterace:} Snadné úpravy a vylepšování návrhů
\end{enumerate}

\subsubsection{Omezení AI}

\begin{itemize}
    \item AI vyžaduje jasné instrukce a kontext
    \item Nutnost manuální kontroly a validace výstupů
    \item Komplexní business logika musí být specifikována člověkem
    \item AI není dokonalá -- některé vztahy vyžadovaly manuální úpravu
\end{itemize}

\subsection{Možná rozšíření}

Systém lze v budoucnu rozšířit o následující funkce:

\begin{enumerate}
    \item \textbf{Multi-user podpora} \\
    Přidání serveru pro sdílení BOM mezi více uživateli v reálném čase
    
    \item \textbf{Historie změn} \\
    Kompletní audit log všech operací s možností rollback
    
    \item \textbf{Alerting} \\
    Automatické notifikace při nízkém stavu zásob
    
    \item \textbf{Barcode generování} \\
    Možnost generovat vlastní QR kódy pro interní součástky
    
    \item \textbf{Statistiky a reporting} \\
    Dashboardy s grafy spotřeby součástek, trendy, predikce
    
    \item \textbf{Mobilní aplikace} \\
    Mobilní verze pro skenování přímo na skladě pomocí telefonu
\end{enumerate}

\subsection{Závěrečné shrnutí}

Tento projekt úspěšně demonstruje komplexní aplikaci objektové metodologie UML na reálný problém skladového managementu elektronických součástek. Všechny vytvořené modely jsou:

\begin{itemize}
    \item \textbf{Konzistentní:} Vzájemně propojené a bez rozporů
    \item \textbf{Kompletní:} Pokrývají všechny aspekty systému
    \item \textbf{Srozumitelné:} Jasně dokumentované a vysvětlené
    \item \textbf{Použitelné:} Slouží jako podklad pro implementaci
\end{itemize}

Systém je navržen s důrazem na:
\begin{itemize}
    \item Jasnou separaci vrstev (UI, Business, Domain, External)
    \item Vysokou kohezi a nízké párování
    \item Snadnou rozšiřitelnost a údržbu
    \item Intuitivní použití pro koncového uživatele
\end{itemize}

\textbf{Aplikace je plně funkční, otestovaná a v produkčním nasazení.}

\vspace{1cm}

\begin{center}
\textit{Tento projekt byl vytvořen s využitím umělé inteligence (GitHub Copilot) \\
pro demonstraci efektivity AI-asistovaného softwarového inženýrství.}
\end{center}

\end{document}
